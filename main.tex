% 独自のコマンド

% ■ アブストラクト
%  \begin{jabstract} 〜 \end{jabstract}  :日本語のアブストラクト
%  \begin{eabstract} 〜 \end{eabstract}  :英語のアブストラクト

% ■ 謝辞
%  \begin{acknowledgment} 〜 \end{acknowledgment}

% ■ 文献リスト
%  \begin{bib}[100] 〜 \end{bib}


\newif\ifjapanese

\japanesetrue  % 論文全体を日本語で書く

\ifjapanese
  %\documentclass[a4j,twoside,openright,11pt]{jreport} % 両面印刷の場合。余白を綴じ側に作って右起こし。
  \documentclass[a4j, 11pt]{jreport}                  % 片面印刷の場合。
  \renewcommand{\bibname}{参考文献}
  \newcommand{\acknowledgmentname}{謝辞}
\else
  \documentclass[a4paper, 11pt]{report}
  \newcommand{\acknowledgmentname}{Acknowledgment}
\fi
\usepackage{thesis}
\usepackage{ascmac}
\usepackage{graphicx}
\usepackage[dvipdfmx]{color} 
\usepackage{multirow}
\usepackage{url}
\usepackage{float}
\usepackage{pgf-pie}
\usepackage[T1]{fontenc}
\usepackage{textcomp}
\usepackage[hang,small,bf]{caption}
\usepackage[subrefformat=parens]{subcaption}
\usepackage{amsmath}
\usepackage{grffile}
\usepackage[a4paper]{geometry}
\usepackage{listings}
	\usepackage{jvlisting}
\captionsetup{compatibility=false}

\bibliographystyle{junsrt}

% 日本語情報(必要なら)
\jclass  {修士論文}                             % 論文種別
\jtitle    {修士論文テンプレート}    % タイトル。改行する場合は\\を入れる
\juniv    {慶應義塾大学大学院}                  % 大学名
\jfaculty  {政策・メディア研究科}               % 学部、学科
\jauthor  {修士 太郎}                       % 著者
\jhyear  {4}                                   % 令和○年度
\jsyear  {2022}                                 % 西暦○年度
\jkeyword  {キーワード、Latex、SFC、修士論文}     % 論文のキーワード

% 英語情報(必要なら)
\eclass  {Master's Thesis}                            % 論文種別
\etitle    {Masters Thesis Template}      % タイトル。改行する場合は\\を入れる
\euniv  {Keio University}                             % 大学名
\efaculty  {Graduate School of Media and Governance}  % 学部、学科
\eauthor  {Taro Syushi}                           % 著者
\eyear  {2022}                                        % 西暦○年度
\ekeyword  {Keyword, Latex, SFC, Masters Thesis}          % 論文のキーワード


% メタ情報管理

\title{
	\@jtitle\
}

\author{\@juniv\ \@jfaculty\ \@jauthor}
\date{\today}

\graphicspath{{resources}}
 
\begin{document}

% 表紙の出力

\ifjapanese
  \jmaketitle    % 表紙(日本語)
\else
  \emaketitle    % 表紙(英語)
\fi

% アブストラクト

% ■ アブストラクトの出力 ■
%	◆書式:
%		begin{jabstract}〜end{jabstract}	:日本語のアブストラクト
%		begin{eabstract}〜end{eabstract}	:英語のアブストラクト
%		※ 不要ならばコマンドごと消せば出力されない。

% 日本語のアブストラクト
\begin{jabstract}
	日本語のアブストラクトをここに書いていく
\end{jabstract}

% 英語のアブストラクト
\begin{eabstract}
	English Abstract
\end{eabstract}  % アブストラクト。要独自コマンド、include先参照のこと


\tableofcontents  % 目次
\listoffigures    % 表目次
\listoftables    % 図目次

\pagenumbering{arabic}

% 本文

\chapter{第1章 はじめに(序論)}

\section{研究のきっかけ・成り立ち}

\section{研究の目的・意義}

\chapter{研究の背景}

\section{国内外の研究状況}
\section{研究の動向}

\chapter{研究の概要}


\section{研究の概要}

\section{研究の特徴・独創性(他の研究との相違を明記する)}

\section{期待する成果}


\chapter{研究成果}

\section{研究アプローチ・結果の詳細}

\section{作品・プログラム等の説明}

\section{特徴ある研究成果の主張}

\chapter{おわりに(結言)}

\section{今後の課題}


\begin{acknowledgment}
謝辞

	\vspace{\fill}
	\rightline{\today}
	\rightline{\@juniv\ \@jfaculty}
	\rightline{\@jauthor}

\end{acknowledgment}  % 謝辞。要独自コマンド、include先参照のこと
\begin{bib}[100]
	\bibliography{main}
\end{bib}  % 参考文献。要独自コマンド、include先参照のこと
\appendix
\chapter{付録}    % 付録

\end{document}